% !TeX encoding = UTF-8

\documentclass{protokol}

\usepackage{tikz}
\usetikzlibrary{calc}
\usetikzlibrary{arrows}

%====== Units =====
\usepackage{siunitx}
\sisetup{inter-unit-product =\ensuremath{\cdot}}
\sisetup{group-digits = integer}
\sisetup{output-decimal-marker = {,}}
\sisetup{exponent-product = \ensuremath{\cdot}}
\sisetup{separate-uncertainty}
\sisetup{tight-spacing = false}
%\sisetup{scientific-notation = true}
%\sisetup{round-mode=places,round-precision=4}
%\sisetup{evaluate-expression}


%====== Grafy =====
\usepackage{pgfplots}
\pgfplotsset{width=0.8\linewidth, compat=1.17}
\def\plotcscale{0.8}
\usepackage{pgfplotstable}
\usepackage[figurename=Obr.]{caption} % figure caption rename
%====== Rovnice align block ======
\usepackage{amsmath}
\setlength{\jot}{10pt} % rozestup mezi řádky

\graphicspath{ {./img/} }

%====== Vyplňte údaje ======
\jmeno{Jakub Charvot}
\kod{240844}
\rocnik{2.}
\obor{MET}
\skupina{MET/4}
\spolupracoval{Radek Kučera}

\merenodne{10.\,11.\,2022}
\odevzdanodne{24.\,11.\,2022}
\nazev{Nízkofrekvenční zesilovače s OZ}
\cislo{4} %měřené úlohy

\predmet{Analogové elektronické obvody}
\ustav{Ústav mikroelektroniky}
\skola{FEKT VUT v Brně}

\def\para{x+0}
\def\parb{\para-80}


\begin{document}
	%====== Vygenerování tabulky ======
	\maketitle
	%====== Úvodní texty protokolu ======

	\section{Teoretický úvod}
		\begin{figure}[h!]
    \centering
    \includegraphics[width=\textwidth]{schema.png}
    \centering
    \caption{Schémata zapojení.}
    \label{fig:schema}
\end{figure}



\subsection{Napájení OZ}

    Rozlišujeme dva základní typy napájení OZ, symetrické a nesymetrické. Sympetrické napětí očekává kladné a záporné napětí na svorkách OZ a potenciál společné země je vůči nim uprostřed, tedy v nule. K tomuto je zapotřebí využít dva zdroje, což je v praxi obtížné, následné zapojení je ale jednodušší. V případě symetrického napájení je na záporné svorce potenciál společné země, stačí nám tedy jeden zdroj. Pokud ale chceme mít jistotu, že výstupní signál nebude zkreslený, je potřeba toto kompenzovat upravením obvodu na vstupu OZ, viz. Obr.~\ref{fig:schema}a).


\subsection{Vypočtené hodnoty pro úlohu č. 1}
    \subsubsection{Stejnosměrné poměry v obvodu}
        $ U_1=\qty{15}{\volt} $, $ U_2=\qty{7.5}{\volt} $, $ U_3=\qty{7.5}{\volt} $, $ U_4=\qty{7.5}{\volt} $, $ U_5=\qty{7.5}{\volt} $, $ U_6=\qty{0}{\volt} $, $ U_7=\qty{0}{\volt} $


    \subsubsection{Střídavé poměry v obvodu, pro amplitudu generátru $ U_M=\qty{1}{\volt} $ }
        $ u_1=\qty{0}{\volt} $, 
        $ u_2=\qty{1}{\volt} $, 
        $ u_3=\qty{0}{\volt} $, 
        $ u_4=\qty{0}{\volt} $, 
        $ u_5=\qty{-10}{\volt} $, 
        $ u_6=\qty{-10}{\volt} $\\
        $ A_u=\frac{u_{out}}{u_{in}}=\frac{-10}{1}=-10 $

\subsection{Odvození vztahů pro úlohu č. 2 a 3}
    \subsubsection{Sumační zesilovač, Obr.~\ref{fig:schema}b)}
        $ u_1 $, $ u_2 $, $ u_3 $ \dots Napětí na rezistorech orientovaná stejně jako proudy. 

        $$ u_{out} = -u_3 = -i_3R_3 = -(i_1+i_2)R_3 = -(\frac{u_1}{R_1}+\frac{u_2}{R_2})R_3 $$
        Hodnoty rezistorů jsou stejné, takže platí:
        $$ u_{out} = -(u_1+u_2) $$

    \subsubsection{Diferenční zesilovač, Obr.~\ref{fig:schema}c)}
        $ u_3 $, $ u_4 $ \dots Napětí na rezistorech orientovaná stejně jako proudy. 

        $$ u_{out} = u_2-u_3-u_4 =u_2 -(u_2-u_1\frac{R_2}{R_1+R_2})-R_4i_4 =  u_1\frac{R_2}{R_1+R_2}-\frac{R_4}{R_3}(u_2-u_1\frac{R_2}{R_1+R_2}) $$
        $$ u_{out}= 2u_1\frac{R_2}{R_1+R_2} -u_2\frac{R_4}{R_3}$$
        Hodnoty rezistorů jsou stejné, takže platí:
        $$ u_{out}=u_1-u_2$$


		
	% \newpage
	% \section{Výsledky počítačové simulace}
	% 	\subsection{Nesymetrické napájení}
    \begin{figure}[h!]
        \centering
        \includegraphics[width=\textwidth]{microcap/Nesym/1.png}
        \centering
        \caption{Stejosměrný pracovní bod zapojení s nesymetrickým napájením.}
        \label{fig:ns-s-pracBod}
    \end{figure}

    \begin{figure}[h!]
        \centering
        \includegraphics[width=\textwidth]{microcap/Nesym/2.png}
        \centering
        \caption{Střídavé poměry v obvodu, vliv vazebních kondenzátorů na vstupu a výstupu obvodu. Horní průběh -- stejnosměrně posunutý vstup a výstup OZ kvůli kompenzaci nesymetrického napájení.}
        \label{fig:ns-s-prubeh1}
    \end{figure}

    \begin{figure}[h!]
        \centering
        \includegraphics[width=\textwidth]{microcap/Nesym/3-r4-56k.png}
        \centering
        \caption{Změna poměru odporového děliče ($R_4=\qty{56}{\kilo\ohm}$), vyšší stejnosměrné posunutí vede k dosažení saturačního napětí.}
        \label{fig:}
    \end{figure}

    \begin{figure}[h!]
        \centering
        \includegraphics[width=\textwidth]{microcap/Nesym/4.png}
        \centering
        \caption{}
        \label{fig:}
    \end{figure}


\clearpage
\subsection{Sumační zesilovač}
        \begin{figure}[h!]
            \centering
            \includegraphics[width=\textwidth]{microcap/Sumac/1.png}
            \centering
            \caption{Stejnosměrný pracovní bod v zapojení sumačního zesilovače.}
            \label{fig:}
        \end{figure}

        \begin{figure}[h!]
            \centering
            \includegraphics[width=\textwidth]{microcap/Sumac/2.png}
            \centering
            \caption{Simulace součtu dvou vstupních signálů na výstupu OZ, výstup je navíc invertovaný.}
            \label{fig:}
        \end{figure}

\clearpage
\subsection{Rozdílový zesilovač}
    \begin{figure}[h!]
        \centering
        \includegraphics[width=\textwidth]{microcap/Difer/1.png}
        \centering
        \caption{Stejnosměrný pracovní bod v zapojení rozdílového zesilovače.}
        \label{fig:}
    \end{figure}

    \begin{figure}[h!]
        \centering
        \includegraphics[width=\textwidth]{microcap/Difer/2.png}
        \centering
        \caption{Simulace odečtení jednoho vstupního signálu od druhého, signál na výstupu OZ je navíc invertovaný.}
        \label{fig:}
    \end{figure}

    \begin{figure}[h!]
        \centering
        \includegraphics[width=\textwidth]{microcap/Difer/3.png}
        \centering
        \caption{Simulace odečtení dvou stejných signálů (zobrazeny přes sebe), na výstupu je signál nulový.}
        \label{fig:}
    \end{figure}

	% \newpage
	% \section{Měření v laboratoři}
	% 	\subsection{Nesymetrické napájení}
\begin{figure}[h!]
    \centering
    \includegraphics[width=0.8\textwidth]{lab/output1.png}
    \caption{Invertující zesilovač, nesym. napájení -- časový průběh správně zesíleného signálu.}
    \label{fig:lab/output1.png}
\end{figure}

\begin{figure}[h!]
    \centering
    \includegraphics[width=0.8\textwidth]{lab/output2.png}
    \caption{Invertující zesilovač, nesym. napájení -- časový průběh správně zesíleného signálu, měřeno přímo na výstupu OZ, tedy s DC složkou.}
    \label{fig:lab-output2-png}
\end{figure}

\begin{figure}[h!]
    \centering
    \includegraphics[width=0.8\textwidth]{lab/output3.png}
    \caption{Imvertující zesilovač, nesym. napájení -- vyšší amplituda vstupního signálu způsobila dosažení saturace.}
    \label{fig:lab-output3-png}
\end{figure}

\begin{figure}[h!]
    \centering
    \includegraphics[width=0.8\textwidth]{lab/output4.png}
    \caption{lab/output4.png}
    \label{fig:lab-output4-png}
\end{figure}

\begin{figure}[h!]
    \centering
    \includegraphics[width=0.8\textwidth]{lab/output5.png}
    \caption{lab/output5.png}
    \label{fig:lab-output5-png}
\end{figure}

\begin{figure}[h!]
    \centering
    \includegraphics[width=0.8\textwidth]{lab/output6.png}
    \caption{lab/output6.png}
    \label{fig:lab-output6-png}
\end{figure}

\begin{figure}[h!]
    \centering
    \includegraphics[width=0.8\textwidth]{lab/output7.png}
    \caption{lab/output7.png}
    \label{fig:lab-output7-png}
\end{figure}

\begin{figure}[h!]
    \centering
    \includegraphics[width=0.8\textwidth]{lab/output8.png}
    \caption{lab/output8.png}
    \label{fig:lab-output8-png}
\end{figure}

\begin{figure}[h!]
    \centering
    \includegraphics[width=0.8\textwidth]{lab/output9.png}
    \caption{lab/output9.png}
    \label{fig:lab-output9-png}
\end{figure}

		
		
	% \clearpage
	% \section{Závěr}
	% 	U zapojení zesilovače s nesymetrickým napájením jsme nejprve vyhodnocovali stejnosměrný pracovní bod.
	% 	Abychom mohli přenášet kladnou i zápornou periodu signálu, je potřeba tento signál stejnosměrně posunout, aby se ve výsledku i zesílený výstupní signál pohyboval v mezích \qty{0}{\volt} - \(U_{nap} \).
	% 	Tab. \ref{tab:dc-bod} obsahuje porovnání hodnot napětí v jednotlivých uzlech získaných výpočtem, simulací a následně měřením. Tyto hodnoty se téměř neliší a drobná odchylka měřených hodnot je způsobena zejména nedokonalým nastavením napájecího napětí (uzel 1).
	% 	Při změně offsetu např. změnou jednoho z odporů na děliči může dojít k ořezání jedné půlvlny signálu, jak je vidět z obrázků 4 a 13. 

	% 	Dále jsme zkoumali zapojení diferenčního a sumačního zesilovače, odvodili jsme vztah pro výstupní napětí, které by se pro diferenční zesilovač melo rovnat rozdílu obou vstupních. Kromě toho je tedy signál invertovaný. 
	% 	Odečtení signálů jsme pozorovali jak v simulaci (Obr. 9 a 10) tak i v laboratoři (Obr. 14, 15 a 16).

	% 	Funkce sumačního zesilovače je obdobná, odvodili jsme, že na výstupu by měl být invertovaný součet obou vstupních signálů. Simulace součtu signálů se nachází na obrázku 7. V laboratoři jsme zapojení testovali pro dva signály s odlišnou amplitudou i frekvencí (Obr. 17) a také pro součet signálu s nulovým signálem, kdy na výstupu je invertovaná verze prvního vstupního signálu (Obr. 18).

	% 	Všechny průběhy nám vyšly dle očekávání a měření v laboratoři odpovídá předešlým simulacím.
\end{document}